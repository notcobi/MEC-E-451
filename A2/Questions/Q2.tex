\section{}
A crane 100 m tall is loading a container full of feathers and delicate glass figurines weighing 5 metric tons onto a cargo ship. While the container is in the air, there is an emergency shutdown of the crane and the case is left hanging for a short time. During this time, winds cause the arm of the crane to vibrate vertically at a frequency of 4 Hz with an amplitude of 5 cm. The supporting cable has an effective stiffness of $k = \frac{100}{L}$ MN/m, where $L$ is the length of exposed cable in metres (neglect changes in $L$ due to vibration).

\begin{figure}[h]
\centering
\includegraphics[width=0.5\textwidth]{Questions/Figures/q2 problem diagram.png}
\end{figure}
\FloatBarrier
\begin{enumerate}[label=(\alph*)]
    \item (2 pts) Determine the cable length $L$ at which the transmissibility would be exactly 1.
    \item (5 pts) Assuming steady state vibration, determine all possible heights $h$ at which the crate would hit the ground.
    \item (3 pts) After the emergency is dealt with, operations resume as the wind continues to excite the crane arm. The crane needs to lift shipments 80 m high to place it onto the cargo ships. What is the smallest mass of cargo that can be lifted in these conditions without shaking with an amplitude greater than 4 cm? Assume $\omega > p$.
\end{enumerate}
\FloatBarrier
\subsection*{Solution}
\subsection{}
First, we must choose the proper model for the system. A simple undamped spring-mass with base excitation model is appropriate. Assume the weight of the cable is negligible. 

The transmissibility is then
\begin{align*}
    \text{TR} &= \frac{\left(\frac{\omega}{p}\right)^2}{1 - \left(\frac{\omega}{p}\right)^2} \\
\end{align*}
converting $f = 4$ Hz to $\omega = 2\pi f = 8\pi$ rad/sec, we can solve for $p$ when $\text{TR} = 1$
\begin{align*}
    1 &= \frac{\left(\frac{8\pi}{p}\right)^2}{1 - \left(\frac{8\pi}{p}\right)^2} \\
    1 - \left(\frac{8\pi}{p}\right)^2 &= \left(\frac{8\pi}{p}\right)^2 \\
    1 &= 2\left(\frac{8\pi}{p}\right)^2 \\
    \implies p &= 8\sqrt{2} \pi
\end{align*}
Since $p$ is defined as 
\begin{align*}
    p &= \sqrt{\frac{k}{m}}\\
    \implies k &= p^2m \\
    &= (8\sqrt{2}\pi)^2 (5 \times 10^3) \\
    &= 0.64\pi^2 \times 10^6 \text{ N/m} \\
    &= 0.64\pi^2 \text{ MN/m}
\end{align*}
and since $k = \frac{100}{L}$, we can solve for $L$,
\begin{align*}
    \Aboxed{L &= \frac{100}{k} = \frac{100}{0.64\pi^2} = 15.8 \text{ m}}
\end{align*}

\subsection{}
% redoing with an even better method.
The DMF is 
\begin{align*}
    \frac{\mathbb{X}}{\mathbb{X}_0} &= \frac{1}{\bigg| 1 - \left(\frac{\omega}{p}\right)^2 \bigg|} \\
    &= \frac{1}{\bigg| 1 - \frac{m\omega^2}{k} \bigg|} \\
    &= \frac{1}{\bigg| 1 - \frac{mL\omega^2}{100 \times 10^6} \bigg|} \\
    &= \frac{1}{\bigg| 1 - \frac{m(100 - h)\omega^2}{10^8} \bigg|} 
\end{align*}
Since we want the amplitude to be equal to or greater than the height of the crate, we're interested in the regions between resonance and the value where the amplitude is equal to the height of the crate.
\begin{figure}[h]
    \centering
    \includegraphics[width=0.5\textwidth]{Questions/Figures/q2 dmf.png}
    \caption{Operating range for b)}
\end{figure}
\FloatBarrier
First, let use an inequality past resonance where the DMF is
\begin{align*}
    \frac{1}{\frac{m(100 - h)\omega^2}{10^8} - 1} &> \frac{h}{a} \\
    1 &> \frac{h}{a} \left(\frac{m(100 - h)\omega^2}{10^8} - 1\right) \\
    1 &> \frac{h}{a} \left(\frac{m(100 - h)\omega^2 - 10^8}{10^8}\right) \\
    1 &> \frac{100m\omega^2h - m\omega^2h^2 - 10^8h}{10^8a} \\
    0 &> \frac{(-m\omega^2)h^2 + (100m\omega^2 - 10^8)h - 10^8a}{10^8a} \\
    0 &> (-m\omega^2)h^2 + (100m\omega^2 - 10^8)h - 10^8a 
\end{align*}
Substituting in the given values,
\begin{align*}
    0 &> (-5 \times 10^3(8\pi)^2)h^2 + (5 \times 10^3(8\pi)^2 \times 100 - 10^8)h - 10^8 \times 0.05 
\end{align*}
Solving for the zeros,
\begin{verbatim}
syms h
eqn = (-5*10^3*(8*pi)^2)*h^2 + (5*10^3*(8*pi)^2*100 - 10^8)*h - 10^8*0.05 == 0;
h = solve(eqn, h);
h = double(h)

>> h =
        0.0232
        68.3140
\end{verbatim}
So then the inequality holds for 
\begin{align*}
    \frac{h}{0.05} > \frac{1}{\left(\frac{\omega}{p}\right)^2 - 1} \implies 
    \begin{cases}
        0 < h < 0.0232 \\
        68.3140 < h
    \end{cases}
\end{align*}
Next, let us find where the inequality holds below resonance. This is
\begin{align*}
    \frac{1}{1 - \frac{m(100 - h)\omega^2}{10^8}} &> \frac{h}{a} \\
    1 &> \frac{h}{a} \left(1 - \frac{m(100 - h)\omega^2}{10^8}\right) \\
    1 &> \frac{h}{a} \left(\frac{10^8 - m\omega^2(100 - h)}{10^8}\right) \\
    1 &> \frac{10^8h - 100 m \omega^2h + m\omega^2h^2}{10^8a} \\
    0 &> (m\omega^2)h^2 + (10^8 - 100m\omega^2)h - 10^8a 
\end{align*}
Substituting in the given values,
\begin{align*}
    0 &> (5 \times 10^3(8\pi)^2)h^2 + (10^8 - 5 \times 10^3(8\pi)^2 \times 100)h - 10^8 \times 0.05
\end{align*}
Solving for the zeros,
\begin{verbatim}
syms h
eqn = (5*10^3*(8*pi)^2)*h^2 + (10^8 - 5*10^3*(8*pi)^2*100)*h - 10^8*0.05 == 0;
h = solve(eqn, h);
h = double(h)

h =
    -0.0232
    68.3603
\end{verbatim}
So then the inequality holds for
\begin{align*}
    \frac{h}{0.05} > \frac{1}{1 - \frac{m(100 - h)\omega^2}{10^8}} \implies 
    \begin{cases}
        -0.0232 < h 
        h < 68.3603
    \end{cases}
\end{align*}
Combining the inequalities, the possible heights are $\boxed{h = (0, 0.0232] \cup [68.3140, 68.3603] \text{ m}}$.

\subsection{}
Let $h = 80$, and the operating range is $\omega > p$. Then, the DMF is
\begin{align*}
    \text{DMF} &= \frac{1}{\left(\frac{\omega}{p}\right)^2 - 1}
\end{align*}
Since the shaking specification is 4 cm, then
\begin{align*}
    \frac{4}{5} &= \frac{1}{\left(\frac{\omega}{p}\right)^2 - 1} \\
    \frac{4}{5} &= \frac{1}{\left(\frac{8\pi}{p}\right)^2 - 1} \\
\end{align*}
Solving for $p$, 
\begin{verbatim}
syms p
eqn = 4/5 == 1/((8*pi/p)^2 - 1);
p = solve(eqn, p);
p = double(p)

>> p =
    -16.7552
     16.7552
\end{verbatim}
The only physical solution is $p = 16.755$ rad/sec.
By the definition of $p$,
\begin{align*}
    p &= \sqrt{\frac{k}{m}} \\
    &= \sqrt{\frac{100/L \times 10^6}{m}} 
\end{align*}
Solving for $m$,
\begin{align*}
    m &= \frac{100/L \times 10^6}{p^2} \\
    &= \frac{100/20 \times 10^6}{(16.7552)^2} \\
    &= \boxed{17.810 \times 10^3 \text{ kg}}
\end{align*}
