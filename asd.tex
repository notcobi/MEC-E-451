
 

Increasing the Survival and Health of Honey Bees Imported by Air from New Zealand and Australia

PHASE 1 REPORT
Bee Team



 
February 9, 2024

Michael Paradis
Business Owner
Paradis Honey
Box 99
Girouxville, Alberta T0H 1S0

Dear Mr. Paradis,

Bee Team is pleased to present the Phase 1: Design Specification and Project Report for the "Increasing the Survival and Health of Honey Bees Imported by Air from New Zealand and Australia" project. This report includes:
•	Project scope
•	Design specifications
•	Specification matrix
•	Preliminary project plan
A total budget of \$61,254 has been estimated for the project. The engineering design cost is \$60,200, composed of 660 junior engineering hours and 12 intermediate engineering hours. Additionally, an approximate cost of the product solution has been estimated to be \$1,054.

Three design concepts will be presented in Phase 2, submitted on March 8, 2024. Finally, a report detailing final design and analysis will be presented on April 11, 2024.
Reach out to the project team David at 780-123-4567 or david@asd.com for any questions or concerns. We look forward to working with you to develop a solution that meets your needs. 

Sincerely,
Bee Team


Enclosure

CC: Dr. John Doucette, Faculty Advisor, University of Alberta
Dr. Tetsu Nakashima, Course Coordinator, University of Alberta

 


 
Table of Contents
List of Figures	iii
List of Tables	iii
1	Introduction	1
2	Design Specifications and Objectives	2
2.1	Key Design Specifications	2
2.2	Design Objectives	2
2.3	Business Side	2
2.4	Table of governing design standards, codes, and regulations	2
3	Specification matrix	4
4	Project Management	5
4.1	Project Deliverables	5
4.2	Cost Estimates	5
5	Conclusion	7
References	8
Appendices	9
Appendix A: Project Proposal	9
Appendix B: Project Management	12
Appendix C: Gantt Chart	12
Appendix D: Team Charter	12



 
List of Figures

List of Tables

Word Count:




 
 
1	Introduction
 
2	Design Specifications and Objectives
% The objective of the design is to develop a ventilation system that will increase the survival and health of honey bees imported by air from New Zealand and Australia. The design will be implemented between 
The objective of the design is to develop a ventilation system on a PMC pallet that will increase the survival and health of honey bees imported by air from New Zealand and Australia. Bee team will provide the mechanical design, analysis of the ventilation system, and component selection. Electrical design, control systems, and programming are not within the scope of the project. 

A drawing package as well as a model of the design will be provided to the client, allowing implementation of the design upon project completion.

% Dry ice will be used to keep the temperature range between 5-20 C, and the ventilation system will provide sufficient airflow to keep the CO2 concentration below 11\% saturation. 

% The design will be implemented between the bee packages on the PMC pallets. 

% The design must provide sufficient ventilation to keep the temperature range between 5-20 C, CO2 concentration below 11\% saturation, and a flow strong enough to "blow out a candle". The design must also be easy to install, operate, and maintain, and must be stackable and portable. The design must be able to operate for 15 hours, withstand ambient cargo hold conditions, and remain securely attached to the PMC during airline transport. The mass of the unit


2.1	Key Design Specifications
%Go over the ‘Core design specs”, and other CRITICAL parameters (ones ranked 3-4)
Team bees has identified specifications for the ventilation system through discussions with the client, requirements of the tranport industry, and the needs of the bees. Core design specification are defined as the most critical parameters that the design must meet. 

2.1.1 Thermal Performance
The design solution must keep the temperature range between 5-20 C. This is a critical design parameter because overheating of bees occurs at temperatures above 20C.

2.1.2 CO2 Concentration
The design solution must provide sufficient ventilation to keep the CO2 concentration below 11\% saturation. Bees are sensitive to CO2 and can suffocate at high concentrations.

2.1.1 International Air Transport Association (IATA) Standards
The design must conform to IATA standards for the transport of live animals. The total mass of batteries on a PMC pallet must be less than 35 kg for cargo airlines, and less than 5 kg for passenger airlines. 

2.1.2 Operation Runtime
The design must also be able to operate for 13 hours, plus a safety factor of 2 hours that considers delays in the loading, unloading, and transport process.

2.1.3 Dimensions and Installation
The design must conform to the configuration of bee packages on the PMC pallets. This limits the device size to be 10'' wide, 10'' long, and X high. The design must also operable by a single person and easily disassemble. 

% Unit must conform to both 4x5' wooden pallet and 8x10' PMC pallet geometry. This limits device size to be less than ~10" wide for ventilating 'PMC core cross' area, and 3" wide for inbetween standard 4x5' wooden pallet rows.



% If using UN3480/UN3481 Lithium batteries, the battery mass is limited to 5kg for passenger airlines, or 35kg for cargo airlines. The design must conform to these standards to be allowed on the airline's cargo hold. This is based on  International Aviation Transport Association standards.
%The unit must keep the temperature range to 5-20 C. This is a critical design parameter because overheating of bees occurs at temperatures above 20C and would invalidiate the 
%Sufficient ventilation must be provided to keep CO2 concentration  below 11% saturation. Each bee enclosure is 9" x 6" x 14", which 432 packages must be supplied with adequate ventilation.


% These include the temperature range, CO2 concentration, and the flow of the ventilation system. The temperature range must be maintained between 5-20 C, as bees are sensitive to temperature and can overheat at temperatures above 20C. The CO2 concentration must be kept below 11\% saturation, as bees are sensitive to CO2 and can suffocate at high concentrations. The flow of the ventilation system should be strong enough to "blow out a candle". These specifications are critical to the design as they directly impact the health and survival of the bees during transport.


2.2	Design Objectives

2.3	Business Side
In the proposed cooling system design, there will be four units with each unit consisting of a fan and a battery to be mounted on each side of the pallet.
As specified by the client, the budget for the design is limited to $7,500, with a total project cost estimate of $1054. This estimate includes $54 for the cost of four Pano-Mount fans. This company was chosen as it offers compact dimensions for installation situations and has high durability which is important for the design of the ventilation system. The estimate also includes the cost of Ionic lithium 12V 20Ah batteries at $220 per unit. These specific batteries are chosen because they are 70% lighter and have a usable capacity of 99% compared to 50% for traditional lead-acid batteries. Furthermore, mounting brackets for each unit costs $30 per unit. 
These costs are estimated with the goal that each unit should be easy to manufacture, install and transport with the most economical solution. Return on Investment will involve an anticipation of long-term cost savings and benefits once the cooling system is successfully implemented. Accurate ROI figures may be determined in the following phases of the project where sufficient data can be analyzed from the proposed design.
2.4	Table of governing design standards, codes, and regulations
-Animal Health Codes (CAN+ NZ+AUS)
-ANSI vibration????
- IATA Standards (this is the REAL STUFF that they use in airlines)

IATA Lithium Ion guidelines:
https://www.iata.org/contentassets/05e6d8742b0047259bf3a700bc9d42b9/lithium-battery-guidance-document-2020.pdf
IATA Portable Electronic Devices (PED), Active Cooling on airlines on planes:
https://www.iata.org/contentassets/05e6d8742b0047259bf3a700bc9d42b9/lithium-battery-guidance-document-2017-for20pharma-en.pdf


 
3	Specification matrix

 
4	Project Management
4.1	Project Deliverables
The project has been broken down into three phases. The table shows the details of the deliverables of those phases and their respective due dates. The Gantt Chart in Appendix C shows a comprehensive version of the project deliverables. These headings must be consistent with Gantt Chart.
Phase	Deliverable	Deadline
1: Problem Definition, Design Specification and Project Plan	Phase 1 Report:
•	Design Specification Matrix
•	Project Schedule
•	Estimated Cost Analysis	February 9
2: Conceptual Design	Phase 2 Report:
•	Design Evaluation Matrix
•	Design Concept
•	Conceptual Design Calculations	March 8
3: Detail Design	Design Conference:
•	Final Design Poster
•	Design Presentation
Phase 3 Report:
•	Design Compliance Matrix
•	Detailed Design Calculations
•	Detailed Design Drawings	April 4


April 11
4.2	Cost Estimates
As part of project planning, the group has set up an estimated time and cost for completing the project deliverables. The table below shows the estimated engineering hours required for each phase and the respective costs per hourly rate. The rate for a junior engineer is $90/hr., and the rate for an intermediate engineer is $150/hr. 
Phase	Junior Engineer (hr)	Intermediate Engineer (hr)	Cost ($)
1	110	2	10,200
2	225	4	20,850
3	310	5	28,650
Presentation	15	1	1500
Total	660	12	60,200

The total estimated cost for the project deliverables is expected to be $60,200 with a total of 660 junior engineer hours and 12 intermediate engineering hours. As Phase 1 was completed, the actual junior engineering hours were tallied up to 105 hours, which is very close to the estimated hours, while the intermediate engineering hours were the same. The difference in junior hours is a result of overestimating the time that may be used for research time on preliminary design concepts. 


 
5	Conclusion


 
References
 
Appendices

Appendix A.	Project Proposal
 
Appendix B.	Full Design Specification Sheet
Item #	Description 	Value	Specification	Design Authority	Importance (1-4)
1.0	Handling	 	 	 	 
1.1	Installation of Device	-	Device needs to be installable given the geometry of the packages. Should be installed before mesh.	Team	3
1.2	Stackability/Portability	-	Units should be easy to disassemble and store so that shipping individual units should be as easy as possible. Stackability synergizes well with this concept.
It should be easy to stack units of the design together so that they can be shipped back to New Zealand with ease.	Client	2
1.3	Operator Usage	-	Device must be operable by a single person/person loading the pallets into the cargo hold. Should be easy to turn on/off	Team	1
					
2.0	Transport	 	 	 	 
2.1	Loading Procedure
 (NZ to CAN)	-	"Air New Zealand do have some very specific requirements for handling shipments of live bees; including only loading in one hold, and having an empty position next to the units that the bees are loaded on to promote airflow around the unit." The loading procedure impacts the transport process.	Standard	2
2.2	Loading Procedure
 (domestic)		TBD from CargoJet/Air Canada	Standard	2
2.3	Operational Hours	13 hours (+2 hour F.O.S)	Once set, the system must be able to continuously operate for 15 hours. This comes from a 13 hour long flight, plus a safety factor of 2 hours that considers loading/unloading process.	Client	3
2.4	Secure Attachment	-	Device must remain securely attached to PMC during airline transport.	Team	3
2.5	Operating Conditions	-	The device should be designed to withstand ambient cargo hold conditions	Client	3
					
3.0	Dimensions and Mass	 	 	 	 
3.1	Overall Unit Mass	10kg/unit	The mass of a unit servicing a 8x10' PMC pallet of packaged bees should be below 10 kg. The ventilation specifications rank higher than a specific weight limit so this a relatively low priority.	Team	2
3.2	Battery Mass	<5kg for Passenger
<35 kg for Cargo	If using UN3480/UN3481 Lithium batteries, the battery mass is limited to 5kg for passenger airlines, or 35kg for cargo airlines. The design must conform to these standards to be allowed on the airline's cargo hold. This is based on  International Aviation Transport Association standards.	IATA Standard	4
3.3	Maximum Dimensions	10" wide	Unit must conform to both 4x5' wooden pallet and 8x10' PMC pallet geometry. This limits device size to be less than ~10" wide for ventilating 'PMC core cross' area, and 3" wide for inbetween standard 4x5' wooden pallet rows.	Client	3
					
4.0	Core Performance Details	 	 	 	 
4.1	Thermal Performance	5-20 C	The unit must keep the temperature range to 5-20 C. This is a critical design parameter because overheating of bees occurs at temperatures above 20C and would invalidiate the 	Client	4
4.2	CO2	11% saturation	Sufficient ventilation must be provided to keep CO2 concentration  below 11% saturation. Each bee enclosure is 9" x 6" x 14", which 432 packages must be supplied with adequate ventilation.	Client	4
4.3	Flow		Qualitatively, the flow should be strong enough "to blow out a candle". CO2/Temperature will be a more driving factor.	Client	2
4.4	Battery Discharge Temperature	minimize	Battery selection needs to be carried out so that heat discharged from battery does not worsen the effects of heat accumulation in the bees transport problem. Heat specs must be considered when selecting batteries.	Team	3
					
5.0	Misc Performance Details	 	 	 	 
5.1	Maintenance Requirements 		Device should have low maintenance requirements for ease of use.	Team	1
5.3	Vibration/Stress onto Bee Packages		The unit should not create excessive vibration or stress where it is installed. Creep and Fatigue must also be within reasonable means. More testing/calculations need to be done on the bee package boxes to ensure compliance to this item.	Team	
5.4	Environmental Impact		Consideration of the environmental impact of the device and its components. The client has not mentioned environmental impacts likely due to the scale being more on the prototype side.	Team	2
5.5	Durability		The unit life should withstand multiple cycles, meaning the unit should itself withstand the various operational stresses it falls under.	Client	2
5.6	Reliability		The unit should operate reliably. 
For example, wire in parallel instead of series to ensure that failure in one component doesn't cause failure to the entire unit.	Client	2
					
6.0	Safety	 	 	 	 
6.1	Fire Hazards		Device must operate in cargo hold conditions without posing a fire hazard. Electrical systems need consideration to avoid shorting or overheating.		3
6.2	Mechanical Safety		Fan Blades should not actively injure humans/bees		3
					
7.0	Project Management	 	 	 	 
7.1	Budget	7500$	The budget per PMC servicing unit must fall within the budget of 7500$	Client	3
7.2	Simplicity	-	The device should be simple to design and operate as to keep the scope in reasonable bounds.	Team	1

Appendix C.	Project Management

Appendix D.	Gantt Chart

Appendix E.	Team Charter


