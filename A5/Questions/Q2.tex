\section{}
\textit{Briefly describe the role of each component in determining the note that a string produces from a vibrational standpoint:}

\subsection{The fretboard}
The fretboard provides a surface for the guitarist to press down the strings against, effectively shortening the length of the vibrating portion of the string. By pressing the string against different frets, the guitarist changes the length of the vibrating portion, thereby altering the frequency of the produced note.

\subsection{The tuning pegs}
The tuning pegs are used to adjust the tension of the strings. Tightening or loosening the strings with the tuning pegs changes their tension, which in turn alters their fundamental frequency when plucked.

\subsection{The bridge}
The bridge of the guitar anchors the strings at the other end of the instrument. It also transmits the vibrations of the strings to the guitar body, contributing to the resonance and overall sound of the instrument.

\subsection{The frets}
The frets are unevenly spaced along the guitar neck to account for the logarithmic nature of musical intervals. The spacing between frets decreases as you move up the neck because the frequency ratio between notes is not linear. The spacing is designed according to the principles of equal temperament to ensure that each fret corresponds to the correct pitch on the musical scale.

\subsection{Bending a note}
Bending a note involves applying pressure to the string with one or more fingers and then moving the string across the fretboard horizontally, effectively increasing the tension and stretching the string. This action increases the pitch of the note being played. From a vibrational standpoint, bending a note alters the tension and length of the vibrating portion of the string, which in turn changes the fundamental frequency and pitch of the produced sound. Additionally, bending can introduce subtle variations in the harmonic content of the sound, adding expressiveness and character to the note.