\section{}
\textit{In this lab, the apparatus is a simple platform suspended by springs as shown above. When modelling a stiffness/elastic element as a spring, it is typically assumed that the spring provides a stiffness in only the axial direction (ie. in the $z$ – direction).}

\begin{enumerate}[label=(\alph*)]
    \item How much would the natural frequency of vibrations in the vertical direction increase if the stiffness of each spring is doubled?
    \item The springs are manufactured such that they are also able to resist lateral forces (ie. in the $x$ and $y$ directions). If the springs have both an axial and lateral stiffness, determine how many degrees of freedom the system has and state each degree of freedom.
\end{enumerate}
\subsection{}
First let the springs be $k$. Since they are in parallel,
\begin{align*}
    k_{\text{eff}} &= 4(k) = 4k \\
    p_1 &= \sqrt{\frac{4k}{m}} = 2\sqrt{\frac{k}{m}}
\end{align*}
Now, let all the springs be $2k$. Then,
\begin{align*}
    k_{\text{eff}} &= 4(2k) = 8k \\
    p_2 = \sqrt{\frac{8k}{m}} &= 2\sqrt{2}\sqrt{\frac{k}{m}}
\end{align*}
So the natural frequency would increase by a factor of 
\begin{align*}
    \Aboxed{\sqrt{2}}
\end{align*}

\subsection{}
If the springs can move in the $x$, $y$, and $z$ directions, then the system has 6 degrees of freedom:
\begin{enumerate}
    \item Displacement in the $x$ direction
    \item Displacement in the $y$ direction
    \item Displacement in the $z$ direction
    \item Rotation about the $x$ axis
    \item Rotation about the $y$ axis
    \item Rotation about the $z$ axis
\end{enumerate}